\documentclass[10pt,a4paper]{article}
\usepackage[utf8]{inputenc}
\usepackage{amsmath}
\usepackage{amsfonts}
\usepackage{amssymb}
\usepackage{makeidx}
\usepackage{graphicx}
\usepackage{lmodern}
\author{Yannick Reiß}
\title{MPT2}
\begin{document}

\section*{Aufgabe 1}
\subsection*{a)}
MPROM (Maskenprogrammiertes ROM)

\subsection*{b)}
Bitleitungen: 8
Wortleitungen: 4

\subsection*{c)}
5 Bits, von denen zwei die Zeile und drei die Spalte festlegen.

\subsection*{d)}
Im Baustein können 32 Bit gespeichert werden

\subsection*{e)}
\begin{itemize}
\item[Adresse] Wert
\item[01100] 0
\item[01101] 0
\item[01110] 0
\item[01111] 0
\item[10000] 1
\item[10001] 0
\end{itemize}

\section{Aufgabe 2}
\subsection*{a)}
Wortbreite: 4 Bit
Bei 2M
Adressanschlüsse $= 4 \cdot 2^{20} = 4197304$

\subsection*{b)}
Die Zeitspanne heißt Erholzeit. Die Zeitspanne kommt durch die Dauer der Entladung nach einem Zugriff.

\subsection*{c)}
Speicherkapazität:\\
$4 \cdot 2048 \cdot 1024 = 8MBit = 1MByte$\\
\\
Die Zeilen werden mit $log_2(2048) = 11$ Bit angegeben.\\
Die Spalten werden mit $log_2(1024) = 10$ Bit angegeben.\\
Damit lassen sich jeweils 4 Bit ansprechen.\\
\\
Abstände Zeile / Spalte:\\
$\frac{64ms}{2038}$\\
$\frac{64ms}{1024}$\\
\\

\subsection*{d)}
Daten werden auf mehrere Bänke verteilt, damit die Wahrscheinlichkeit für ein Ansprechen der Bank während der Precharge Zeit minimiert.


\section*{Aufgabe 3}

\subsection*{a)}
$3^2 + 7^2 = 9 + 49= 58$ Adressen\\

\subsection*{b)}




\end{document}